%-------------------------------------------
\againframe<2>{RNAseqWF_diapo}
%-------------------------------------------
\begin{frame}[containsverbatim]
\frametitle{Data and command line}
%-------------------------------------------
\begin{exampleblock}{Data}
-g genome sequence acces (including extention .fna, .fasta)\\
-a genome annotation acces (inluding extention .gff)\\
-d RNAseq sample prefix\\
next args: RNAseq sample prefix, no .fastq.gz extention
\end{exampleblock}
\begin{exampleblock}{Bash command line}
\begin{lstlisting}
FAIR_initial_script.sh -g ../O.tauri_genome.fna -a ../O.tauri_annotation.gff -d ../ SRR3099585_chr18 S*86_chr18 S*87_chr18 S*97_chr18 S*98_chr18 S*99_chr18
\end{lstlisting}
\end{exampleblock}
\begin{exampleblock}{Script in 3 main blocks}
\begin{lstlisting}
1) while getops do ... done   
2) for sample in $* ; do ... done   
3) creation of the result file, counts.txt, with paste, awk, and sed bash commands
\end{lstlisting}
\end{exampleblock}
\end{frame}
%-------------------------------------------
\begin{frame}[containsverbatim]
\frametitle{Complete bash script, 1}
%-------------------------------------------
\begin{exampleblock}{getops block}
\begin{lstlisting}
while getopts g:a:d: flag do
        case $flag in
            g)  genome=$OPTARG
                echo genome is $genome ;;
            a)  annots=$OPTARG
                echo annotation is $annots ;;
            d)  rnadir=$OPTARG
                echo RNAseq path is $rnadir ;;
            :)  echo "L'option $OPTARG requiert un argument"
                exit 1 ;;
            \?) echo "$OPTARG : option invalide"
                exit 1 ;;
        esac
   done
shift $(( OPTIND - 1 ))  # shift past the last flag or argument
echo samples are $*
\end{lstlisting}
\end{exampleblock}
\end{frame}
%-------------------------------------------
\begin{frame}[containsverbatim]
\frametitle{Complete bash script, 2}
%-------------------------------------------
\begin{exampleblock}{for block}
\begin{lstlisting}
nbs=0;
for sample in $* ; do 
   nbs=$(expr ${nbs} + 1)
   echo traitement of sample ${sample}
   # --------- quality control of reads
   if [ ! -d FastQC ]; then
       mkdir FastQC
   fi
   fastqc --outdir FastQC ${rnadir}${sample}.fastq.gz > FastQC/${sample}.log 2>&1
   #---------- reads mapping
   if [ ! -d Bwt2_index ]; then
       mkdir Bwt2_index
       bowtie2-build ${genome} Bwt2_index/tauri > Bwt2_index/Bwt2_index.log 2>&1
   fi
\end{lstlisting}
\end{exampleblock}
\end{frame}
%-------------------------------------------
\begin{frame}[containsverbatim]
\frametitle{Complete bash script, 3}
%-------------------------------------------
\begin{exampleblock}{for block, continuation}
\begin{lstlisting}
   bowtie2 -x Bwt2_index/tauri -U ${rnadir}${sample}.fastq.gz -S ${sample}.sam > ${sample}_bowtie2.log 2>&1
   #---------- selection and format modification
   samtools view -b ${sample}.sam -o ${sample}.bam
   samtools sort ${sample}.bam -o ${sample}_sort.bam
   samtools index ${sample}_sort.bam
   #---------- counting of mapped reads by gene
   featureCounts -t gene -g ID -a ${annots} -s 1 -o ${sample}_ftc.txt ${sample}_sort.bam > ${sample}_ftc.log 2>&1
done
\end{lstlisting}
\end{exampleblock}
\begin{exampleblock}{Count table block}
\begin{lstlisting}
paste *_ftc.txt > ftc_tmp.txt
awk -v nb=${nbs} -v col=7 'BEGIN{FS="\t"}{ctmp=$1; for(i=col;i<=nb*col;i=i+col){count=sprintf("%s\t%s",ctmp,$i);ctmp=count};print count}' ftc_tmp.txt | sed 1d > counts.txt
\end{lstlisting}
\end{exampleblock}
\end{frame}
%-------------------------------------------
\begin{frame}[containsverbatim]
\frametitle{Exercise 2}
%-------------------------------------------
Continue the snakefile of the previous exercise in order to replace the bash script. \\
We will:
\begin{exampleblock}{Objectives}
\begin{itemize}
    \item add a configuration file
    \item use a builtin snakemake function to get filenames of the input RNAseq data
    \item add rules to replace the mapping, formatting, counting, and counts aggregating steps of the bash script
\end{itemize}
\end{exampleblock}
\begin{exampleblock}{ex2$\_$o1.smk}
\begin{lstlisting}
cp ex1_o7.smk ex2_o1.smk
\end{lstlisting}
\end{exampleblock}
\end{frame}
%-------------------------------------------
\begin{frame}[containsverbatim]
\frametitle{getops block}
%-------------------------------------------
\begin{exampleblock}{Shell script}
\begin{lstlisting}
while getopts g:a:d: flag do
        case $flag in
            g)  genome=$OPTARG 
            ...
\end{lstlisting}
\end{exampleblock}
We will use a configuration file:
\begin{exampleblock}{Objective 1}
Add a configuration file, named \verb|RNAseq.yml|, containing both the genome sequence and the annotation files manes, and the access to the Data directory. \\
In the snakefile, change the configured variables (ex. replace \verb|Data/| and \verb|genome| by their \verb|config[]| values). The Python strings concatenation is \verb|+|\\
Then, run snakemake with the \verb|--configfile| option.
\end{exampleblock}
\end{frame}
%-------------------------------------------
\begin{frame}[containsverbatim]
\frametitle{Adding a configuration file}
%-------------------------------------------
\begin{exampleblock}{RNAseq.yml}
\begin{lstlisting}
genome:
  O.tauri.fna
annots:
  O.tauri.gff
RNAseqDir:
  Data/
\end{lstlisting}
\end{exampleblock}
\begin{exampleblock}{Snakemake run}
\begin{lstlisting}
rm -Rf FastQC/ Result/ Tmp/ Logs/ ; snakemake -s ex2_o1.smk --configfile RNAseq.yml
\end{lstlisting}
\end{exampleblock}
\end{frame}
%-------------------------------------------
\begin{frame}[containsverbatim]
\frametitle{for block}
%-------------------------------------------
\begin{exampleblock}{Shell script}
\begin{lstlisting}
nbs=0;
for sample in $* ; do 
   ...
done
\end{lstlisting}
\end{exampleblock}
To manage all *.fastq.gz files in a directory, use the \verb|glob_wilcards()| function. In \verb|ex2_o2.smk|, replace the SAMPLES definition by:
\begin{exampleblock}{ex2$\_$o2.smk}
\begin{lstlisting}
SAMPLES, = glob_wildcards(config["dataDir"]{sample}.fastq.gz")
\end{lstlisting}
\end{exampleblock}
and run snakemake.
\end{frame}
%-------------------------------------------
\begin{frame}[containsverbatim]
\frametitle{Quality control, fastqc}
%-------------------------------------------
\begin{exampleblock}{}
\begin{lstlisting}
if [ ! -d FastQC ]; then
   mkdir FastQC
fi
fastqc --outdir FastQC ${sample}.fastq.gz > FastQC/${sample}.log 2>&1
\end{lstlisting}
\end{exampleblock}
No more need to test the existence of a directory, it is created as needed.
\begin{exampleblock}{rule fastqc:}
This rule is already present in the snakefile
\end{exampleblock}
\end{frame}
%-------------------------------------------
\begin{frame}[containsverbatim]
\frametitle{Reads mapping, bowtie2}
%-------------------------------------------
\begin{exampleblock}{}
\begin{lstlisting}
if [ ! -d Bwt2_index ]; then
   mkdir Bwt2_index
   bowtie2-build ${genome} Bwt2_index/tauri > Bwt2_index/Bwt2_index.log 2>&1
fi
bowtie2 -x Bwt2_index/tauri -U ${sample}.fastq.gz -S ${sample}.sam > ${sample}_bowtie2.log 2>&1
\end{lstlisting}
\end{exampleblock}
2 rules: \verb|genome_bwt2_index| (cf. previous ex.) and \verb|bwt2_mapping|
\begin{exampleblock}{ex2$\_$o3.smk, rule bwt2$\_$mapping:}
\begin{lstlisting}
  output: "results/{sample}.sam"
  input: config["dataDir"]+"{sample}.fastq.gz",
         expand("Tmp/Otauri.{ext}.bt2", ext=BIDX)
  log: "Logs/{sample}_bwt2_mapping.log"
  shell: "bowtie2 -x Tmp/Otauri -U {input[0]} -S {output} 2> {log} "
\end{lstlisting}
\end{exampleblock}
\end{frame}
%-------------------------------------------
\begin{frame}[containsverbatim]
\frametitle{samtools}
%-------------------------------------------
\begin{exampleblock}{Shell script}
\begin{lstlisting}
samtools sort -O bam -o ${sample}_sort.bam ${sample}.sam
samtools index ${sample}_sort.bam
\end{lstlisting}
\end{exampleblock}
\begin{exampleblock}{ex2$\_$o4.smk, rule sam2bam$\_$sort:}
\begin{lstlisting}
  output:
    bam="Result/{sample}_sort.bam",
    bai="Result/{sample}_sort.bam.bai"
  input: "Tmp/{sample}.sam"
  log:
    sort="Logs/{sample}_sam2bam_sort.log",
    index="Logs/{sample}_bam2bai.log"
  shell: 
    "samtools sort -O bam -o {output.bam} {input} 2> {log.sort} ;"
    "samtools index {output.bam} 2> {log.index}"
\end{lstlisting}
\end{exampleblock}
\end{frame}
%-------------------------------------------
\begin{frame}[containsverbatim]
\frametitle{FeatureCount}
%-------------------------------------------
\begin{exampleblock}{Shell script}
\begin{lstlisting}
featureCounts -t gene -g ID -a ${annots} -s 1 -o ${sample}_ftc.txt ${sample}_sort.bam > ${sample}_ftc.log 2>&1
\end{lstlisting}
\end{exampleblock}
\begin{exampleblock}{ex2$\_$o5.smk, rule counting:}
\begin{lstlisting}
  output: "Tmp/{sample}_ftc.txt"
  input:
    bam="Result/{sample}_sort.bam",
    annot=config["dataDir"]+config["annots"]
  log: "Logs/{sample}_counts.log"
  shell: "featureCounts -t gene -g ID -a {input.annot} -s 1 -o {output} {input.bam} &> {log}"
\end{lstlisting}
\end{exampleblock}
\end{frame}
%-------------------------------------------
\begin{frame}[containsverbatim]
\frametitle{Counts matrix creation}
%-------------------------------------------
\begin{exampleblock}{Shell script}
\begin{lstlisting}
paste *_ftc.txt > counts_tmp.txt
awk -v nb=${nb_sample} 'BEGIN{FS="\t"}{count_tmp=$1; for(i=7;i<=nb*7;i=i+7){count=sprintf("%s\t%s",count_tmp,$i);count_tmp=count};print count}' counts_tmp.txt | sed 1d > counts.txt
\end{lstlisting}
\end{exampleblock}
\begin{exampleblock}{Hint}
Create 2 rules to manage some files aggregation to one result file:
\begin{itemize}
    \item rule \verb|extract_counts|: extract geneID and counts in individual files
    \item rule \verb|matrix_counts|: paste these files
\end{itemize} 
\end{exampleblock}
\end{frame}
%-------------------------------------------
\begin{frame}[containsverbatim]
\frametitle{Counts matrix creation}
%-------------------------------------------
\begin{exampleblock}{ex2$\_$o6.smk, 2 rules:}
\begin{lstlisting}
rule matrix_counts:
  output: "Result/counts_matrix.txt"
  input: countfile=expand("Tmp/{sample}_ftc7.txt", sample=SAMPLES), geneID=expand("Tmp/{sample}_ftc1.txt", sample=SAMPLES)
  log: "Logs/matrix_counts.log"
  shell: """cp {input.geneID[0]} Tmp/ftc_geneID.txt ; paste Tmp/ftc_geneID.txt {input.countfile} > {output} &> {log}"""

rule extract_counts:
  output: col7="Tmp/{sample}_ftc7.txt", col1="Tmp/{sample}_ftc1.txt"
  input: "Tmp/{sample}_ftc.txt"
  log: "Logs/{sample}_extract_counts.log"
  shell: """cut -f 7- {input} | sed 1d > {output.col7} &> {log} ; cut -f 1 {input} | sed 1d > {output.col1} """
\end{lstlisting}
\end{exampleblock}
\end{frame}
%-------------------------------------------
\begin{frame}[containsverbatim]
\frametitle{DESeq2}
%-------------------------------------------
The DESeq2 step is the statistical analysis. From the count matrix, the statistical analysis is managed by a non parallelizable R script, DESeq2
\end{frame}
%-------------------------------------------
\begin{frame}[containsverbatim]
\frametitle{Last challenge}
%-------------------------------------------
\begin{exampleblock}{}
Clean, delete and re-run !
\begin{lstlisting}
cp ex2_o8.smk RNAseq_analysis.smk
cp ex2_o1.yml RNAseq_analysis_smkEnv.yml
rm -Rf FastQC/ Results/ Logs/ Tmp/ 
snakemake -s  RNAseq_analysis.smk --configfile RNAseq_analysis_smkEnv.yml
\end{lstlisting}
\end{exampleblock}
\end{frame}
%-------------------------------------------
\begin{frame}[containsverbatim]
\frametitle{Bonus}
%-------------------------------------------
\begin{exampleblock}{Add a help rule}
\url{https://lachlandeer.github.io/snakemake-econ-r-tutorial/self-documenting-help.html#a-help-rule}
\end{exampleblock}
\end{frame}
